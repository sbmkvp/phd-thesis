\section{Discussion and Research Gaps}

From the above literature search we can conclude there is a considerable opportunity in the application of mobile phone based data to be used to measure small-scale spatio-temporal dynamics of human activity and potential for research gaps in the following areas:
Hyper-local population estimation This research is at the limits of acceptable mobility research in the context of privacy concerns i.
 between the explosion of usage of smartphones and implementation of MAC randomisation.
The availability of identifiable device information has led to a considerable effort in urban mobility studies and there are still opportunities in real time small-scale population measurement and estimation.
Detailed WiFi data, combined with other data sources such as public transport, weather etc.
along with interpolation and modelling techniques can produce detailed spatio-temporal estimates of population at local level.
De-Anonymisation of devices The increase in privacy concerns and the normalisation of device level security measures such as MAC address randomisation has questioned the basic assumptions about the reliability of data collected through mobile devices.
There is a definite research need and interest in finding a method to uniquely identify devices with sufficient confidence while not deciphering any personal data from such analysis.
Device classification There has been a significant volume of studies in the classification of space based on the temporal patterns in the human activity around them and classification of human activity based on the space it is clustered around.
But there is a research opportunity in looking at these devices closely, especially their temporal patterns around the point of interest to make inferences about the nature of the device.
This idea is closely related to de-anonymisation discussed about but rather than trying to find unique users or devices from a set of signals, we try to find the groups of similar devices and connecting these groups to meaningful typologies.

Sensor catchment and flow Having access to a broader sensor data, we can cross-sectionally see the distribution or occurrence of a unique MAC address / device and determine the influence of a sensor.
This can be approached in two ways: the first being the pedestrian flow approach where we can model the movement of pedestrian based on these relationships and the second is treating these relationships as network and detecting communities in them.
In addition to these, there is a definite opportunity for research in methods for visualising high frequency, hyper-local pedestrian data within the limits of cognition of the viewer.

In terms of technology, which can be used for data collection at local level, the summary of the advantages of each is given below,

	Interpolation
	Bespoke Systems
	Cellular Network
	GPS 
	WiFi
	Coverage
	-
	Local
	City to National
	Local to Global
	Local
	Certainty
	Very Low
	High
	Medium
	High
	Medium
	Independence*
	Low
	Very High
	Low
	Medium
	High
	Intrusiveness
	Low
	Medium
	High
	High
	Medium
	Granularity
	Very Low
	Very High
	Medium
	High
	High
	Ease of Collection
	Medium
	Low
	Medium
	Low
	High
	Scalability
	Medium
	Low
	High
	Medium
	High
	Privacy Risk
	Low
	Medium
	High
	High
	Medium
	

Table 2.
: Evaluation of advantages of different approaches that can be used for data collection.
*independence from secondary data collected by a third party.
From Table 2.1 we can see that though it poses some risk on privacy of participants and some uncertainty regarding range, WiFi offers the best possible technology for data collection through mobile devices at smaller scales.
