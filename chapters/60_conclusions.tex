\chapter{Discussion and Conclusions}

Sentient technologies make measurement of the human activities that are the life blood of the smart city possible.
Yet the data that they harvest are frequently relevant only to the sub-groups within society that avail themselves of particular goods and services – such as social media applications, transport modes or retail offers.
In each of these cases, it is necessary to remember that the resulting data are by-products of consumer transactions, and will as a consequence, only pertain to users of the relevant goods or services.
If the smart city is to be socially inclusive, it therefore follows that sentient data must represent entire populations, whether by design or by triangulation with external, population wide, sources.
This is a non-trivial task, since the ebbs and flows of smart device-enabled citizens rarely pertain to any clearly defined population in either administrative or functional terms \citep{massam1975}.

Our objective here has been to collect, rather than re-use, data on smart city functioning, by recording Wi-Fi probes and ultimately reconciling them with manual counts in order to infer ambient populations.
The internal validation methodology set out in the technical sections of this paper, allied to external validation from pedestrian counts, renders the method inclusive and robust when recording activity levels in retail centres in real time.
We have described the collection and processing of a novel consumer Big Dataset that enables valid measures of levels of footfall activity which has been scaled across a wide network of sensors \citep{cdrc2018}.
 In both conceptual and technical terms, it illustrates the ways in which passively collected consumer data can be ‘hardened’ to render them robust and reliable by using related procedures of internal and external validation.

Internal validation addresses the issues of screening out device probes that do not indicate footfall, and the further screening of device probes to ‘fingerprint’ the effects of MAC randomization.
It is important to note that the filtering process work based solely on the information present in the probe requests and their temporal distribution.
This ensures that although the mobile devices were uniquely identified, there was no further personal data generated by linking the probe requests to the users of the mobile devices.
This method essentially gave us a way to estimate the footfall in real-time without identifying or tracking the mobile devices themselves.
External validation then entailed reconciling adjusted counts with the footfall observed at sample locations.
This procedure makes it possible to generalise from locations at which manual footfall surveys are conducted to all others in the system, and to develop a classification of device locations that are more or less susceptible to noise generation.

This Wi-Fi based footfall counting methodology offers a large number of applications and benefits for real time spatial analysis.
Since Wi-Fi based sensors are inexpensive and the data model is scalable, it is possible to use this methodology for a large network of sensors to gather granular data on pedestrian footfall.
A snapshot showing a week’s worth of precise footfall in area around Charring cross, London is shown in Figure \ref{main_study_counts} in order to demonstrate the potential for such a dataset.
Projects such as SmartStreetSensors \citep{cdrc2018}, may utilise this methodology to overcome the challenges introduced by the implementation of MAC address randomisation.

The vicissitudes of MAC randomisation, and the provisions of privacy legislation such as EU General Data Protection Regulations mitigate against tracking individuals across the smart city using this approach.
 This can be modelled using agent-based methods \citep{heppenstall2011}, however.
In our own research we have also begun to link store time-lagged till receipts to footfall, and have used such data to better understand the dwell times that characterise such different retail uses as stores with window displays and fast food restaurants.
 Such analysis not only provides a more nuanced picture of movement through retail areas, but also enables valorisation of micro sites within retail centres.
In the UK, for example, this is of immediate practical importance in evaluating business rates on properties, and has still wider implications for the setting of retail unit rental values.
There are obvious extensions to understanding the ebbs and flows of activities in the 24-hour smart city such as understanding urban mobility \citep{gariazzo2019} and conceptualising them with a people dimension \citep{nam2011}.

More broadly still, extensions to this strand of smart city research are likely to seek to differentiate the quality of different elements within footfall according to mission e.. travel to adjacent workplace zones, leisure, etc. and personal characteristics such as spending power. In this respect, future research may not only simulate linkage of harmonised footfall counts between sensor locations, but also link these in turn to disaggregate origin-destination matrices for bikeshare and other public transport modes. Our own investigations will consider these and other challenges to understanding the functioning of the sentient city.
