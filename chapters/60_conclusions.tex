%==============================================================================%
\chapter{Discussion and Conclusions}
%==============================================================================%
In the past 30 years there have been an immense change in the way data concerting distribution and dynamics of human population are generated, collected and consumed.
Rather than being a top-down, structured designed endeavour, data generation and collection has become a bottom-up procedure where data were created as a result of day to day activities of people and are collected, cleaned, and  aggregated into information.
There have been a significant volume of research on identifying such data sources and use them for various purposes in both academia and industry.
As these data sources got more distributed and comprehensive, the concern to protect user privacy increased as well.
This thesis aims to work in between these two the areas of research and the corresponding opportunities present in them.
This thesis identifies Wi-Fi probe requests as a source of data from which information on ambient population and behaviour - especially footfall - could be extracted, and solves the problem of inferring accurate footfall information without using personally identifiable information of the users.

%------------------------------------------------------------------------------%
\section{Summary of Findings}
%------------------------------------------------------------------------------%

From the systematic literature review of around 350 academic publications, it was concluded that Wi-Fi is the most suitable candidate for the technology that can be used to collect data on human presence and movement at a national level.
It was found to be the scalable, cheap, universal, and easy way to collect large amounts of granular such data without depending on any other infrastructure.
The only shortcomings of projects using Wi-Fi technology are their inherent uncertainties and the leakage of MAC addresses - a globally unique, personally identifiable information which could be related back to the users relatively easily.
Finally, two potential areas of research with opportunities for further study were identified:  Creating a standardised, cleaned, accurate and reliable footfall or ambient population from the Wi-Fi probe requests, and solving the specific issue of MAC address randomisation while cleaning and filtering Wi-Fi probe requests. 
After studying the Wi-Fi specification to get a overall outline of the structure of Wi-Fi probe request, a set of small initial experiments were designed and executed to know more about them.
It was found that the number of probe requests and the unique MAC addresses collected by a Wi-Fi sensor is far greater than the number of mobile devices in the immediate vicinity.
The signal strength and sequence numbers are some of the important information that are present in the probe requests.
Alongside these experiments a longer pilot study was also conducted to result in three distinct datasets.
The small experiments collected in-depth data on probe requests at small areas for short intervals.
The pilot study covered 5 locations across central London collected data for over two and a half months.
Finally the Smart Street Sensor project which collected small set of data from probe requests at 1000 locations continuously for over 3.5 years.

Before moving to cleaning and processing the data, this research undertook a comprehensive look at the nature of these datasets within the context of 'Big data and Big data tools' so that a framework for evaluating the 'bigness' of the datasets can be devised.
With such a framework the Wi-Fi datasets were examined in all of their dimensions and found to be 'Medium data' at best.
A review of big data tools was also carried out and the tools suitable for the Wi-Fi dataset were picked out and combined together to form a bespoke 'Medium data toolkit' for processing the Wi-Fi data as efficiently as possible.

From the initial exploration, the major uncertainties identified in the data which needs to be solved by the cleaning and processing procedures are, range of the Wi-Fi sensor, differing frequency at which mobile devices emit probe requests, MAC address randomisation which masks the devices unique identification, increasing mobile devices ownership in the population over long-term, missing data from the failures in the sensors and collisions of MAC addresses when they are anonymised using cryptographic hashing.
The collisions in hashed MAC addresses were found to be rare and inconsequential.The uncertainty regarding the range of the Wi-Fi sensor was found to cause noise in the data from outside the field of measurement and was solved by filtering out probe requests with low signal strength.
This definition of 'Low' signal strength could be deduced dynamically for each location at each time interval using one dimensional clustering algorithms.
The 'k-means' algorithm was found to be best suited for this purpose.
The randomisation of MAC addresses lead to over-estimation of number of devices from set of probe requests while the uneven frequency of probe requests emitted by the mobile devices prohibit a simple universal factor for converting number of probe requests to number of devices.
It was found that this uncertainty can be solved using a novel graph based methodology which uses the sequence numbers in the packets rather than the MAC address to uniquely identify the devices.
When the sequence number is not available the uncertainty can be reduced for an interval by looking at the ratio of number of probe requests to the number of mobile devices in the probe requests without randomised addresses in that interval.
These methods along with manual calibration were found to reduce the error in the estimation of footfall from Wi-Fi probe requests to almost 10\% at locations with ideal conditions.

In addition to the above data cleaning techniques other processing were done to the probe requests dataset to remove further uncertainties.
The missing data in the dataset could be interpolated using a Kalman smoothing based method for short term or a seasonally decomposed method for long term.
Finally, the increasing mobile ownership was adjusted using manual counts for short term intervals and using a weekly adjustment factor 0.2\% for long term.
Using all these methods for filtering, cleaning and adjusting Wi-Fi data, this thesis finalises a overall data processing pipeline for producing a clean, precise, accurate and continuous data on footfall across retail locations across the UK.

Finally this thesis provides a gallery of examples showing the possible use of such granular and continuous data on footfall on a national level.
The data were used to devise a 'footfall index' at various levels - national, city, area and micro site locations showing how the retail related footfall have been distributed in the UK and how this distribution has been changing over time in high granularity both spatially and temporally.
It has been demonstrated that this information on footfall can be used as a clue for knowing the form and function of a place and trace the changes it has undergone over time as well.
It was also demonstrated two sets of examples, that real-world events could be detected from looking at the anomalies in the footfall volumes at locations.
Finally, it was also demonstrated that such detailed and continuous footfall volume information at locations could be used to predict or estimate flow of pedestrians between them by just looking at the changes in these volumes thus providing a way to understand the pedestrian flow in cities without actually tracking individuals.


%------------------------------------------------------------------------------%
\section{Research Questions}
%------------------------------------------------------------------------------%


%------------------------------------------------------------------------------%
\section{Further Work}
%------------------------------------------------------------------------------%


%Concluding remarks

