\section{Uncertainties in Data}

Having set up the data collection process, organised the data for quick and easy retrieval and satisfied with the consistency of the data collection infrastructure, the next step is the identification of further uncertainties in the data and formation of informed assumptions to move forward with the analysis.
The major source of uncertainties we encounter and assumptions we undertake are as follows:

Range of the sensor: Since the strength of the signal from a mobile device to the WiFi access point depends on various factors such as distance between them, the nature and size of obstructions between them, interference from other electromagnetic devices etc., the exact delineation of the range of the sensors is almost to impossible.
Moving forward in the research we assume that the range of the sensor is equal in all directions and is linearly indicated by the RSSI (received signal strength indicator) reported by the mobile devices in range.

Probe request frequency: The frequency of probe requests generated by device varies widely based on the manufacturer, operating system, state of the device and the number of access points already known to the device as illustrated in Figure 3.9 and 3.10 (Freudiger, 2015).
These requests are also generated in short bursts rather than at regular intervals.
Moreover android devices send probe requests even when the WiFi is turned off.
With the large number of different devices available, it is impossible to predict and create a general model for this probing behaviour.
For simplicity, we assume that for a probe request received which has a MAC address with a known OUI, there is a corresponding device present within the range of the sensor at that time interval, irrespective of the number of such requests received in the mentioned interval.
Essentially we are just looking for unique MAC addresses within a time period rather than the total number requests made by them.

MAC address collisions: From the initial analysis we have observed that there are few instances of MAC address collisions reported where a device known to be in some place has been reported somewhere else.
This might be occurring due to rogue MAC randomisation by certain devices and the hashing procedure done at two different places.
Due to the negligible volume of such collisions (~2\%), for the purpose of this report, we ignore these collisions and treat all distinct hashed MAC addresses with know OUID to be the same device.
