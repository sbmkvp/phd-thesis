\section{Initial Experiments}

With our theoretical understanding of the Wi-Fi standard and its capabilities, we move on to looking at the Wi-Fi landscape in real-world.
We achieve this by designing small independent experiments where we record the Wi-Fi probe requests within controlled conditions along with the knowledge of the ambient population of the field of measurement. 
We then look at the collected probe requests, examine them in detail to look at their properties, aggregate them to footfall counts and compare them with the real-world counts to get a overall idea of how well they translate into real counts.
The aim of these experiments to know more about the probe requests data and pick out the uncertainties and opportunities present in them.
The objectives here are,

\begin{enumerate}
  \setlength{\itemindent}{2em}
  \itemsep-0.25em
   \item Design a simple method to collect probe requests.
  \item Select locations with different levels of complexity.
  \item Collect real-world data through manual counting.
  \item Analyse the probe requests to extract useful information.
\end{enumerate}

\subsection{Experiment Design}

The first step was to design a simple method to collect Wi-Fi probe requests.
We accomplish by utilising the application - \textit{tshark} \cite{wireshark2} on a regular laptop.
First we put the Wi-Fi module of the laptop in `Monitor mode' where it behaves as a wireless access point.
Then we run tshark to collect the data in CSV format using the following command under a shell.

\begin{minted}{bash}
#! /bin/bash
tshark \
  -I -i en0 \
  -T fields \
  -E separator=, \
  -E quote=d \
    -e frame.time \
    -e frame.len \
    -e wlan_radio.signal_dbm \
    -e wlan_radio.duration \
    -e wlan.sa_resolved \
    -e wlan.seq \
    -e wlan.tag.length \
    -e wlan.ssid \
  type mgt subtype probe-req and broadcast
\end{minted}

The counting of pedestrians next to the sensor is done manually by the surveyor.

\subsection{Isolated Environment}
First setup using the laptop and wireshark in the living room.

\subsection{University Hall}
Second set of experiments in UCL cloisters.

\subsection{Oxford Street}
To start, we designed a small pilot study to validate the filtering and clustering methodology against the scale and complexity of data collected in an open public area such as a retail high street.
We also aimed to find the algorithm which was best suited for the classification of signal strengths as 'low' and 'high' in order to filter out the background noise.
The data was collected at Oxford Street, London on 20 December 2017 from 12:30 to 13:00 hrs, Wi-Fi probe requests were collected using the sensor described in Section and pedestrian footfall was manually recorded using the Android app - Clicker bala2018clicker.
Being located at one of the busiest retail locations in the United Kingdom, the Wi-Fi sensor captured approximately 60,000 probe requests during the half hour period; 3,722 people were manually recorded walking on the pavement during that time.
The surveyor positioned himself at the front of a store while carrying the sensor in a backpack and counted people walking by the store on the pavement (3m wide approximately) using a mobile phone.
The sensor was kept as close to the store window as possible, and the manual count was done as a cordon count in front of the store.

\subsection{Summary and Discussion}
