\section{Initial Experiments}

First setup using the laptop and wireshark in the living room.
Second set of experiments in UCL cloisters.
To start, we designed a small pilot study to validate the filtering and clustering methodology against the scale and complexity of data collected in an open public area such as a retail high street.
We also aimed to find the algorithm which was best suited for the classification of signal strengths as 'low' and 'high' in order to filter out the background noise.
The data was collected at Oxford Street, London on 20 December 2017 from 12:30 to 13:00 hrs, Wi-Fi probe requests were collected using the sensor described in Section and pedestrian footfall was manually recorded using the Android app - Clicker bala2018clicker.
Being located at one of the busiest retail locations in the United Kingdom, the Wi-Fi sensor captured approximately 60,000 probe requests during the half hour period; 3,722 people were manually recorded walking on the pavement during that time.
The surveyor positioned himself at the front of a store while carrying the sensor in a backpack and counted people walking by the store on the pavement (3m wide approximately) using a mobile phone.
The sensor was kept as close to the store window as possible, and the manual count was done as a cordon count in front of the store.


