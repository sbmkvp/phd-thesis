%==============================================================================%
\chapter{Introduction}
%==============================================================================%

Our understanding of the form and function of cities and the built environment has evolved significantly since the early twentieth century.
What started as a field of research focused on the physical form of spaces and places, later moved towards modelling them as a function of the population that lives in them.
Rather than viewing the built environment as infrastructure which need to be built, maintained and managed independently, cities have increasingly been viewed as the manifestation of the distribution and dynamics of the population embedded within them.
The field was further broadened in the later part of the twentieth century to include the economic and social activity which happens within the fabric of the built environment.
Moreover, with the dawn of the information age around the turn of the millennium, the built environment can now be viewed as the tangible result of information exchange; where cities can be seen as high density clusters of information exchange, in addition to as places with a concentration of physical infrastructure such as buildings and roads.
This information revolution has not only changed researchers' understanding of the underlying forces of the built environment, but has also changed how they approach the task of measuring, analysing, modelling and managing it.
The information revolution has provided researchers with numerous new technologies, methodologies, and tools.
Perhaps most importantly however, is the unprecedented availability of comprehensive, granular data generated from fundamental functions of the built environment, such as human mobility, social interaction and economic activity.
Availability of these data and tools has turned numerous disciplines upside down resulting in research which tackles problems using a bottom-up, `data first' approach, rather than a more traditional top-down `systems' approach. 

We are currently in an age of `data deluge' where the amount of data generated in the world far exceeds our capacity to analyse and derive insights from them.
This deluge of data has accelerated to such an extent that 90\% of all the data ever  generated in the world has been created in the last 2 years \cite{ibm2016}.
With the popularisation of wearable technologies  and the `internet of things', this trend is not expected to change any time soon.
Moreover, many day-to-day  activities of people such  as banking, bill payments, public transport ticketing, taxi hire, social communications, and fitness tracking have been digitised and are generating large amounts of unstructured data as a consequence.
As such, collecting data for some types of quantitative research has changed from a highly structured, designed endeavour to a low cost, scraping activity from data repositories heretofore relatively unused in terms of research beyond the purpose for which the data were initially collected.
Most of the data collection activity has also become `passive', i.e. collected without any effort from the participants.
This has vastly increased the capacity of the data collection process, which has led to the emergence of `big data' and consequentially, to the need for advanced and automated data-mining techniques to extract value from these vast datasets.
The above two phenomena – the attempt to model the physical environment as a function of information exchange, and the unprecedented availability of data - has led to a significant volume of research wherein various data sources have been utilised to understand a variety of aspects of the built environment.
For example, functional regions of a country has been derived from call detail record data \cite{ratti2010}, and population and demography have been studied through social media data such as that derived from Twitter.

This frenzy of data generation and use is not without pitfalls.
One of the major disadvantages in the attempts to repurpose the data is the risk  to the privacy of the users whose data is being collected and analysed.
With personal mobile devices becoming mainstream, almost every data point generated has a person behind it.
The rush into the information age and the use of social media platforms has happened at a much faster rate than understandings of the ramifications to privacy of the participants could be properly understood.
Even when the data collected does not contain personal information, most datasets can reveal personal and potentially sensitive information when linked with other sources of data.
For example, anonymised cycle ride trajectories might not  be interesting information on their own, but when combined with other datasets such as taxi trips and payment information, the data can disclose the identity and residences of the people the data is about.
This has prompted major concerns and backlashes from users and regulators in the past decade.
These concerns are addressed in industry as well as research using both technology and regulation.
From the technology perspective, all the stakeholders who generate, collect, or use the data try to use cryptography to anonymise, obscure, or encrypt any personal information as much as possible.
In terms of regulation, legislation efforts such as the General Data Protection Regulation (GDPR) have been introduced to influence the behaviour of these stakeholders by introducing comprehensive rules and punitive measures for non-compliance.
Though both these approaches ultimately try to protect user's privacy and personal information, they also pose one of the greatest challenges to research which uses passively collected user data.
In the next 5 years, it can be expected that every freely available data source will be protected from the unfettered use which we see today.
Wherever this protection is not possible, it can be expected that the data would be obscured or anonymised in order to remove any risk to the privacy of users, thus making it imperative that researchers adapt to these changes by looking for ways to overcome the challenges posed by them.

In addition to privacy concerns, this deluge of data introduces significant technological challenges as well. 
Both academia and industry have produced extensive `big data' research which develops the theory, methods and tools to tackle the challenges posed by such large assemblages of data, in order to derive meaningful insights from them.
This `big data' research promises to solve a lot of the technological and logistical challenges incurred in many disciplines, but not without significant additional overheads in terms of cost and resources.
In the case of research projects, blindly adopting the `big data' methods without consideration, has the potential to cause more problems than advantages.
The discipline of Geography, especially geographic information systems and science, has a long tradition of dealing with large datasets from the inception of the field, and the recent deluge of data has caused issues due to the complexity, latency and lack of structure of these new datasets, rather than their sheer volume.
Hence, it is extremely important to be mindful while adopting the contributions from `big data' discourse for research so that the solutions are implemented where the actual problems are located.
There needs to be careful consideration when choosing or designing the methods, tools and frameworks which are used to address the unique requirements of the new data sources.
Moreover, there needs to be an inquiry into a framework for how these considerations are identified and addressed.

In this context, the research described in this thesis works on the opportunities and pitfalls presented above in the following ways: by first describing the collection of large volumes of passively generated data, then by solving the uncertainties in the data which arises due to their high variability and the mechanisms designed to protect the privacy of the users, and finally, there is an analysis of the data to produce useful information regarding the distribution and dynamics of footfall in the country.
The thesis starts with a broad and systematic literature survey on the topic of `distribution and dynamics of human activity' in Chapter \ref{chapter:literature}.
In this chapter, major themes of research and their evolution in the past 30 years are identified along with the development of technologies which were employed.
The literature review resulted in the identification of the best possible data source for further research, along with opportunities available for further research.
Having identified Wi-Fi as one of the most promising technologies for research, Chapter \ref{chapter:collection} explores Wi-Fi specification in detail, especially the `probe request' mechanism.
In addition to studying the standards and specification used to identify relevant data, the chapter also discusses the design and implementation of a series of small experiments to capture and analyse data in the real-world.
Three sets of initial experiments were conducted and results from the experiments were used to conduct a longer and broader `pilot study' which collected data from locations across London.
The chapter also introduces the `Smart Street Sensors'(SSS) project - a national project which collects Wi-Fi data at a large number of retail locations.
The chapter concludes with a detailed evaluation of all the data collected from the experiments and the SSS project, in terms of the bias, noise and uncertainties present in them.
Chapter \ref{chapter:processing} deals with processing the Wi-Fi data to remove the identified uncertainties in order to produce `clean' and continuous information on the volume of footfall at the corresponding locations.
The emphasis on not using personal data, or methods which can potentially reveal personal information, is firmly held throughout the chapter.
In section \ref{section:toolkit}, a framework for evaluating the `bigness' of the data is discussed, and a `data toolkit' for processing them is subsequently devised.
In section \ref{section:processing}, methods to clean the data into a realistic estimate of footfall are discussed.
In section \ref{section:pipeline}, both the `data toolkit' and methods are combined to architect a `data pipeline' which digests the continuous stream of data from the SSS project into meaningful footfall numbers efficiently.
Chapter \ref{chapter:applications} details a variety of applications of the research across four major themes: an index for footfall across United Kingdom, the detection of events using changes in the volume of footfall, an estimation of the flow of pedestrians between locations derived from the changes in footfall volumes, and the identification of the nature and relationship between places along with possibilities for further research.

The potential of creating such detailed, long-term, national-level footfall data as produced by this research is immense.
Such information can be one of the major components in building a `smart city', where the availability of detailed, real-time data on the built environment and its use is vital.
It can also help us in our pursuit to accomplish a real-time census of people and their movement in the city.
It can not only provide us with snapshots of the state of  retail areas, but also help in measuring, modelling and manipulating them in real-time as a dynamic system which respond to interventions.
We can even link these footfall data to other sources of data such as commercial consumer datasets and public transport statistics, in order to build a comprehensive picture on the health and efficiency of city-wide systems.
Availability of such datasets can revolutionise academic research in fields such as urban planning, public policy and urban management, whereby the effect of interventions could be objectively measured and analysed.
Although this research did not try to explore the applications of this footfall data in detail, it hopes to serve as a solid basis for further studies in a various academic disciplines such as geography, business management, risk management, spatial analysis and computer science, which can employ the data to either derive insights about locations and context, or use the data as a reference/training source for validating methods and tools.
Availability of such national level data on footfall volumes spanning continuously over years can also have a substantial impact on industries such as retail, transportation, real estate and information technology.
As this research has a significant bias towards retail locations, the outputs can especially be of immense value for various stakeholders in the retail industry such as \textit{Retailers} who can get detailed information on when and where their customers shop which can lead to more efficient business operations, \textit{Customers} who can be informed on the popularity of places and when to visit them, \textit{Landlords} who can achieve a way to objectively evaluate their properties' values based on their location and also time, and \textit{Local Authorities} who can be enabled to monitor and manage the health of their retail areas over longer periods of time.
%------------------------------------------------------------------------------%
