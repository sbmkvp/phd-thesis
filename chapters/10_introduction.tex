\chapter{Introduction}


We talk about the theory of cities and built environment. We start from how
these have been perceived as function of the form and gradually changed to
people, activity, economy and information. Built environment is manifestation of
information exchange that happens in them. We talk about the change in theory
regarding this. We talk about how this information exchange has been becoming
more and more open and the opportunity it provides us planners, geographers and
researchers to understand these things better.

Along with the information age there is an explosion of open data. The data
collection has changed from structured high effort activity to low effort
scraping activity. The data generated by scraping is unprecedented and
staggering. Most of the Big-data research has gone into this in the past decade.
Disadvantages of structured data which the unstructured data fills. This is
changing how we view, understand and experience the world. Some of these
datasets fall into this unique medium size category as well which are neither
big data nor trivial. There is a need for methods and tools to collect, convert
and use these data.

Talk about the ubiquity of the mobile technology. Everyone has a device which
connects them with world wirelessly. Major ones cellphone and Wi-Fi Wi-Fi is
uniquely placed in between Cellphone and Bluetooth. The design of Wi-Fi gives us
amazing opportunity. This has been done before for the past decade by loads. The
privacy advocacy has become a new thing. The change is from both ends.
Collectors are regulated, cellphones are getting sneakier. Need for method to
collect data and analyse it without compromising on privacy. The conversion of
this unstructured data into something tangible and measurable is not a trivial
problem. There are loads of such data and measurements. Examples - banking vs
economic activity, oyster card data vs movement.

The potential use of such information is immense. Give examples of smart city
paradigm and connected city where real time census is possible. We can not only
take snapshot of the state of the city, we can record and understand the built
environment as living, breathing organism. The insights we get by combining this
information with other similar info is more than sum of their parts. It can
revolutionise understanding , planning, policy etc, urban management and finally
industry such as retail, transportation etc. city mapper, sharing economy etc.
